\documentclass[a4paper,10pt]{article}
\usepackage[margin=2.5cm, nohead]{geometry}
\usepackage{graphicx}
\usepackage{verbatim}
\newcommand{\Ardrone}{Ar Drone$^{\copyright}$ }

\title{Autonomous Flight With the \Ardrone}
\author{Maarten Inja \& Maarten de Waard\\\small 5872464 \& 5894883}

% The presentatie 

\begin{document}
\maketitle

\section{Introduction}

\subsection{The \Ardrone}
The \Ardrone is an over WiFi remote controlled quadrocopter that has several onboard sensors:
\begin{itemize}
	\item One vertical camera, pointing downwards
	\item One horizontal camera, pointing forward 
	\item Ultrasound altimeter, to measure the altitude
    \item 3 axis accelerometer (measures propellor acceleration)
    \item 2 axis gyrometer 
    \item 1 yaw precision gyrometer
\end{itemize}
Furthermore, it has an onboard computer system running Linux. 


\subsection{Our Goal}
Summer-IMAV 2011 Indoor competition, some sub-tasks of the exploration challenge:
\begin{itemize}
    \item Pick-up Object
    \item Exit Building
    \item Release Object
\end{itemize}

\section{Controlling the \Ardrone}

\subsection{SDK}
Software Development Kid

\subsection{Extending C with Python}
It's awesome possum

\section{Methods Used}

\subsection{Finding The Object}

\subsection{Recognizing The Object}

\subsection{Picking Up The Object}


% In the presentation we had sections for problems and results, but I guess 
% we should just put that stuff in the other chapters (and as Arnoud said
% make it more like "we tried this and this but discarded that because that 
% and that)


\section{Results}

\section{Future Work}

\end{document}































